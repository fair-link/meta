\section{Posicionamento}

\subsection{Descrição do problema}
Segundo a Euronews, "cerca de 20\% dos alimentos produzidos na Europa acabam em aterros ou são usados para alimentar gado, o que custa à economia 140 mil milhões de euros por ano." de acordo com uma notícia publicada em Outubro de 2020  ( https://pt.euronews.com/2020/10/12/tecnologias-e-metodos-para-combater-desperdicio-alimentar-na-ue). Os supermercados, aonde se situa parte deste desperdício, são organizações que têm maior adopção de sistemas de informação em relação aos pequenos negócios, pelo que a integração dos seus sistemas com um sistema de informação que registe e organize doações de alimentos estará facilitada.\\

Actualmente os grupos de venda a retalho têm departamentos de Responsabilidade Social e encontram-se abertos para aceitar pedidos de doação, mas não existe nenhum processo uniforme para todos os supermercados que facilite a doação de alimentos. Geralmente pedidos de donativo são efectuados pelas associações à sua loja local, pelo que também é difícil para os supermercados terem uma visão global da sua acção neste âmbito.\\

Os produtos em quebra de stock são bons candidatos a serem doados, uma vez que não podem ser vendidos. Constituem um desafio tecnológico interessante, pois a janela temporal reduzida para o seu consumo obriga a que o processo de doação tenha que ser rápido, i.e. no mesmo dia que são identificados.\\

Pretende-se desenvolver uma aplicação que ponha em contacto os supermercados e entidades do sector social, por forma a diminuir o desperdício alimentar, através de doações de alimentos que não podem ser vendidos.\\
\begin{itemize}
	\item Problema - Realizar uma aplicação que permita automatizar doações por parte de supermercados, por intermédio das IPSS
	\item Afecta - Supermercados, IPSS ou outras organizações do sector social, população carenciada
	\item Redução do desperdício alimentar, Combate de desigualdades sociais no acesso à alimentação
	\item Aplicação com possibilidade de converter quebras de stock dos supermercados em doações a indivíduos carenciados.
\end{itemize}

\subsection{Posicionamento do produto}
O posicionamento do produto a realizar é sucintamente descrito na tabella seguinte:
\begin{itemize}
	\item Para - Empregados dos departamentos de Resposabilidade Social de grandes superfícies, Funcionários IPSS, indivíduos carenciados;
	\item Quem disponibiliza - Samuel Costa;
	\item Produto - Sistema de informação para reduzir desperdício alimentar nas grandes superfícies, efectuando doações à população carenciada através de organizações do sector social;
	\item Função - Distribuir produtos em quebra de stock nas grandes superfícies à população carenciada;
	\item Em vez de - Desperdício alimentar; Contactos singulares das associações com as lojas individuais;
	\item O produto proposto - permite registar disponibilidade de donativos, formular propostas de donativo, rastrear levantamentos e entregas de donativos e confirmar doações.
	
\end{itemize}

\clearpage