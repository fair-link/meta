\section{Introdução}
\pagenumbering{arabic}

Neste documento são reunidas, analisadas e definidas as necessidades e as características gerais de um sistema de informação para diminuição do desperdício alimentar nas grandes superfícies, através da doação das quebras de stock ao sector social. A descrição apresentada foca-se nas características requeridas pelo cliente e pelos potenciais utilizadores. Os detalhes do modo como o referido sistema suportará essas características são descritos nos documentos de casos de utilização e de especificação suplementar.

\subsection{Objectivo}
O objectivo desta visão é obter e consolidar conhecimento preliminar acerca da realização de um sistema de informação para diminuição do desperdício alimentar nas grandes superfícies através da doação das quebras de stock a organizações do sector social.

\subsection{Âmbito}
Esta visão insere-se no âmbito da realização do Projecto Individual, no contexto da unidade curricular Engenharia de Software, do Mestrado em Engenharia Informática e de Computadores do DEETC do ISEL.

\subsection{Definições, abreviaturas e acrónimos}
Ver glossário anexo a esta entrega.

\subsection{Referências}
\begin{itemize}
	\item DV-PRA-01 - Luís Morgado, ISEL-DEETC,2008.
\end{itemize}
\subsection{Organização do documento}

O documento está organizado da seguinte forma:
\begin{itemize}
	\item Secção 1: secção de introdução;
	\item Secção 2: secção onde é apresentada a descrição do problema e o posicionamento do produto;
	\item Secção 3: secção onde é apresentada a descrição geral da solução;
	\item Secções 5 a 9: secções onde são descritos requisitos e restrições aplicáveis ao produto;
	\item Secção 10: secção onde é apresentado o estado global da solução.
\end{itemize}

\clearpage