\section{Descrição Geral da Solução}

\subsection{Resumo das Características}

A solução proposta consiste numa aplicação informática que diariamente recebe das grandes superfícies ficheiros com relação de produtos alimentares que estas pretendem doar. O sistema realiza a distribuição dos artigos pelas associaçoes registadas, de acordo com princípios como a proximidade, tamanho e carência da população servida. As associações formulam pedidos previamente preenchidos às grandes superfícies.\\
No mesmo dia, um funcionário da associação (recolhedor) recolhe os donativos no armazém da grande superfície, registando a recolha através da leitura de um QR code e introdução do seu PIN na aplicação. De seguida, são emitidos alertas SMS para os indivíduos/agregados familiares referenciados para que levantem os donativos nas instalações da associação.\\
No levantamento o indivíduo carenciado fornece um código de 6 caracteres presente no alerta SMS e o funcionário de logística da associação introduz esse código no sistema, concluindo a doaçao.\\

\subsection{Assunções e dependências}
O funcionamento do sistema depende da recepção diária de um ou mais ficheiros das grandes superfícies.\\

\subsection{Custo e preço}

O custo será avaliado e discutido posteriormente.

O produto será disponibilizado aos utilizadores gratuitamente.
\subsection{Licenciamento e Instalação}
O produto deverá ter componentes com licença pública e componentes protegidos por direitos de autor. O licenciamento será avaliado posteriormente.

O produto será instalado num provedor de cloud por um elemento da equipa de desenvolvimento.\\

\clearpage